%% BioMed_Central_Tex_Template_v1.06
%%                                      %
%  bmc_article.tex            ver: 1.06 %
%                                       %

%%IMPORTANT: do not delete the first line of this template
%%It must be present to enable the BMC Submission system to
%%recognise this template!!

%%%%%%%%%%%%%%%%%%%%%%%%%%%%%%%%%%%%%%%%%
%%                                     %%
%%  LaTeX template for BioMed Central  %%
%%     journal article submissions     %%
%%                                     %%
%%          <8 June 2012>              %%
%%                                     %%
%%                                     %%
%%%%%%%%%%%%%%%%%%%%%%%%%%%%%%%%%%%%%%%%%


%%%%%%%%%%%%%%%%%%%%%%%%%%%%%%%%%%%%%%%%%%%%%%%%%%%%%%%%%%%%%%%%%%%%%
%%                                                                 %%
%% For instructions on how to fill out this Tex template           %%
%% document please refer to Readme.html and the instructions for   %%
%% authors page on the biomed central website                      %%
%% http://www.biomedcentral.com/info/authors/                      %%
%%                                                                 %%
%% Please do not use \input{...} to include other tex files.       %%
%% Submit your LaTeX manuscript as one .tex document.              %%
%%                                                                 %%
%% All additional figures and files should be attached             %%
%% separately and not embedded in the \TeX\ document itself.       %%
%%                                                                 %%
%% BioMed Central currently use the MikTex distribution of         %%
%% TeX for Windows) of TeX and LaTeX.  This is available from      %%
%% http://www.miktex.org                                           %%
%%                                                                 %%
%%%%%%%%%%%%%%%%%%%%%%%%%%%%%%%%%%%%%%%%%%%%%%%%%%%%%%%%%%%%%%%%%%%%%

%%% additional documentclass options:
%  [doublespacing]
%  [linenumbers]   - put the line numbers on margins

%%% loading packages, author definitions

%\documentclass[twocolumn]{bmcart}% uncomment this for twocolumn layout and comment line below
\documentclass{bmcart}

%%% Load packages
%\usepackage{amsthm,amsmath}
%\RequirePackage{natbib}
%\RequirePackage[authoryear]{natbib}% uncomment this for author-year bibliography
%\RequirePackage{hyperref}
\usepackage[utf8]{inputenc} %unicode support
%\usepackage[applemac]{inputenc} %applemac support if unicode package fails
%\usepackage[latin1]{inputenc} %UNIX support if unicode package fails


%%%%%%%%%%%%%%%%%%%%%%%%%%%%%%%%%%%%%%%%%%%%%%%%%
%%                                             %%
%%  If you wish to display your graphics for   %%
%%  your own use using includegraphic or       %%
%%  includegraphics, then comment out the      %%
%%  following two lines of code.               %%
%%  NB: These line *must* be included when     %%
%%  submitting to BMC.                         %%
%%  All figure files must be submitted as      %%
%%  separate graphics through the BMC          %%
%%  submission process, not included in the    %%
%%  submitted article.                         %%
%%                                             %%
%%%%%%%%%%%%%%%%%%%%%%%%%%%%%%%%%%%%%%%%%%%%%%%%%


\def\includegraphic{}
\def\includegraphics{}



%%% Put your definitions there:
\startlocaldefs
\endlocaldefs

\newcommand{\variantSpark}{{\sc VariantSpark}}
\newcommand{\ARI}{Adjusted Rand Index}


%%% Begin ...
\begin{document}

%%% Start of article front matter
\begin{frontmatter}

\begin{fmbox}
\dochead{Research}

%%%%%%%%%%%%%%%%%%%%%%%%%%%%%%%%%%%%%%%%%%%%%%
%%                                          %%
%% Enter the title of your article here     %%
%%                                          %%
%%%%%%%%%%%%%%%%%%%%%%%%%%%%%%%%%%%%%%%%%%%%%%

\title{VariantSpark: Population Scale Clustering of Genotype Information}

%%%%%%%%%%%%%%%%%%%%%%%%%%%%%%%%%%%%%%%%%%%%%%
%%                                          %%
%% Enter the authors here                   %%
%%                                          %%
%% Specify information, if available,       %%
%% in the form:                             %%
%%   <key>={<id1>,<id2>}                    %%
%%   <key>=                                 %%
%% Comment or delete the keys which are     %%
%% not used. Repeat \author command as much %%
%% as required.                             %%
%%                                          %%
%%%%%%%%%%%%%%%%%%%%%%%%%%%%%%%%%%%%%%%%%%%%%%

\author[
   addressref={aff1,aff2},                   % id's of addresses, e.g. {aff1,aff2}
%   corref={aff1},                       % id of corresponding address, if any
%   noteref={n1},                        % id's of article notes, if any
%   email={jane.e.doe@cambridge.co.uk}   % email address
]{\inits{}\fnm{Aidan R.} \snm{O'Brien}}
\author[
   addressref={aff1},
]{\inits{}\fnm{Neil} \snm{Saunders}}
\author[
   addressref={aff3,aff4},
%   email={john.RS.Smith@cambridge.co.uk}
]{\inits{}\fnm{Fabian A.} \snm{Buske}}
\author[
   addressref={aff1},
   email={Denis.Bauer@CSIRO.au}
%   email={john.RS.Smith@cambridge.co.uk}
]{\inits{}\fnm{Denis C.} \snm{Bauer}}
%%%%%%%%%%%%%%%%%%%%%%%%%%%%%%%%%%%%%%%%%%%%%%
%%                                          %%
%% Enter the authors' addresses here        %%
%%                                          %%
%% Repeat \address commands as much as      %%
%% required.                                %%
%%                                          %%
%%%%%%%%%%%%%%%%%%%%%%%%%%%%%%%%%%%%%%%%%%%%%%

\address[id=aff1]{%                           % unique id
  \orgname{CSIRO, Health \& Biosecurity Flagship}, % university, etc
  \street{11 Julius Av},                     %
  \postcode{2113},                                % post or zip code
  \city{Sydney},                              % city
  \cny{Australia}                                    % country
}
\address[id=aff2]{%
  \orgname{School of Biomedical Sciences and Pharmacy, Faculty of Health},
  \postcode{2308},
  \city{Newcastle},
  \cny{Australia}
}
\address[id=aff3]{%
  \orgname{Cancer Epigenetics Program, Cancer Research Division, Kinghorn Cancer Centre, Garvan Institute of Medical Research},
  \street{384 Victoria St},
  \postcode{2010},
  \city{Sydney},
  \cny{Australia}
}
\address[id=aff4]{%
  \orgname{UNSW Medicine, University of New South Wales},
%  \street{384 Victoria St},
  \postcode{2052},
  \city{Sydney},
  \cny{Australia}
}
%%%%%%%%%%%%%%%%%%%%%%%%%%%%%%%%%%%%%%%%%%%%%%
%%                                          %%
%% Enter short notes here                   %%
%%                                          %%
%% Short notes will be after addresses      %%
%% on first page.                           %%
%%                                          %%
%%%%%%%%%%%%%%%%%%%%%%%%%%%%%%%%%%%%%%%%%%%%%%

\begin{artnotes}
%\note{Sample of title note}     % note to the article
\note[id=n1]{Equal contributor} % note, connected to author
\end{artnotes}

\end{fmbox}% comment this for two column layout

%%%%%%%%%%%%%%%%%%%%%%%%%%%%%%%%%%%%%%%%%%%%%%
%%                                          %%
%% The Abstract begins here                 %%
%%                                          %%
%% Please refer to the Instructions for     %%
%% authors on http://www.biomedcentral.com  %%
%% and include the section headings         %%
%% accordingly for your article type.       %%
%%                                          %%
%%%%%%%%%%%%%%%%%%%%%%%%%%%%%%%%%%%%%%%%%%%%%%

\begin{abstractbox}

\begin{abstract} % abstract
\parttitle{Motivation}: Genomic information is increasingly used in medical practice, giving rise to the need for efficient analysis methodology able to cope with thousands of individuals and millions of variants. The widely used Hadoop MapReduce architecture and associated machine-learning library, Mahout, provide the means for tackling computationally challenging tasks. However, many genomic analyses do not fit the Map-Reduce paradigm. We therefore survey the recently developed {\sc Spark} engine, along with its associated machine learning library, MLlib, which are more flexible in the parallelisation architecture and hence more suitable for population-scale bioinformatics tasks. To do this, we developed an interface from Mahout and MLlib to the standard variant format (VCF), which opens up the usage of advanced, efficient machine learning algorithms to genomic data. 

\parttitle{Results}: We successfully clustered more than 2,500 individuals each having more than 38 Million variants. Both Hadoop and {\sc Spark} have superior performance over traditional approaches. We observe a 50 fold speedup when using the efficient in-memory {\sc Spark} compute engine compared to Hadoop. Furthermore, our implementation achieves a better performance compared to a previously published {\sc Spark}-based genome clustering approach, ADAM.

\parttitle{Availability}: The package is written in Scala and available at https: //github.com/BauerLab/VariantSpark.
\end{abstract}

%%%%%%%%%%%%%%%%%%%%%%%%%%%%%%%%%%%%%%%%%%%%%%
%%                                          %%
%% The keywords begin here                  %%
%%                                          %%
%% Put each keyword in separate \kwd{}.     %%
%%                                          %%
%%%%%%%%%%%%%%%%%%%%%%%%%%%%%%%%%%%%%%%%%%%%%%

\begin{keyword}
\kwd{sample}
\kwd{article}
\kwd{author}
\end{keyword}

% MSC classifications codes, if any
%\begin{keyword}[class=AMS]
%\kwd[Primary ]{}
%\kwd{}
%\kwd[; secondary ]{}
%\end{keyword}

\end{abstractbox}
%
%\end{fmbox}% uncomment this for twcolumn layout

\end{frontmatter}

%%%%%%%%%%%%%%%%%%%%%%%%%%%%%%%%%%%%%%%%%%%%%%
%%                                          %%
%% The Main Body begins here                %%
%%                                          %%
%% Please refer to the instructions for     %%
%% authors on:                              %%
%% http://www.biomedcentral.com/info/authors%%
%% and include the section headings         %%
%% accordingly for your article type.       %%
%%                                          %%
%% See the Results and Discussion section   %%
%% for details on how to create sub-sections%%
%%                                          %%
%% use \cite{...} to cite references        %%
%%  \cite{koon} and                         %%
%%  \cite{oreg,khar,zvai,xjon,schn,pond}    %%
%%  \nocite{smith,marg,hunn,advi,koha,mouse}%%
%%                                          %%
%%%%%%%%%%%%%%%%%%%%%%%%%%%%%%%%%%%%%%%%%%%%%%

%%%%%%%%%%%%%%%%%%%%%%%%% start of article main body
% <put your article body there>

%%%%%%%%%%%%%%%%
%% Background %%
%%
\section*{Background}

Genomic information is increasingly used in medical practice.
A commonly performed task in such applications is grouping individuals based on their genomic profile to identify population association~\cite{Gao2007} or elucidate different haplotype involvement in diseases susceptibility~\cite{Laitman2013}.  
Due to the decreasing sequencing cost it is economical to generate studies with sample sizes previously reserved for larger consortia such as the 1000 genomes project~\cite{1KG2012} or The Cancer Genome Atlas, TCGA~\cite{TCGA2013}. 
At the same time, whole genome sequencing enables the inclusion of rare or even somatic mutations in the analysis, increasing the feature space by orders of magnitude. This drastic increase in both sample numbers and features per sample requires a massively parallel approach to data processing~\cite{Stein2010}. Traditional parallelisation strategies like OpenMPI or hardware accelerators (GPGPU) cannot scale to variable data sizes or require purpose-build hardware.
 
Addressing this issue, {\sc Apache Hadoop MapReduce}~\cite{Borthakur2007} transforms data into 'key-value pairs' that can then be distributed between multiple nodes across a commodity compute-cluster depending on the size of the problem. 
MapReduce approaches are increasingly being used in bioinformatics (for reviews see~\cite{Zou2013, Qiu2010,Taylor2010}). 
This is especially the case for sequence analysis tasks, such as read mapping~\cite{Schatz2009}, duplicate removal~\cite{Jourdren2012}, and variant calling~\cite{Langmead2009, McKenna2010} as well as Genome Wide Analysis Study based tasks~\cite{Huang2013, Guo2014}. 
Apache also developed a machine learning library, Mahout~\cite{Owen2011}, which allows efficient out-of-the-box analysis to be applied to clinical applications, such as medical health records~\cite{Ko2014}.
Unfortunately, the MapReduce paradigm is not always the optimal solution, specifically for bioinformatics or machine learning applications that require iterative in-memory computation. In addition, Hadoop is disk IO intensive, and this can prove to be a bottleneck in processing-speed.

{\sc Apache Spark}~\cite{Zaharia2011} is a more recent compute engine, which overcomes many of Hadoop's limitations. 
One of the main benefits is that it allows programs to cache data in memory; potentially eliminating, or at least reducing, the bottleneck of disk IO. 
When utilising caching, Apache claim {\sc Spark} to be 100x faster than Hadoop. 
Although {\sc Spark} allows MapReduce-like programs, it does not require programs to exactly model the MapReduce paradigm, which in turn allows more flexibility when designing programs. 
Recognising the capability, the Big Data Genomics (BDG) group has recently demonstrated the strength of {\sc Spark} in a genomic clustering application using ADAM, a set of formats and APIs as well as processing stage implementations for genomic data~\cite{Massie2013}. 
While the speedup over traditional methods was impressive, being limited by constraints within this general genomics framework hampered performance. 

We hence developed a more focused purpose-built application in {\sc Spark} to perform genomic clustering of individuals. 
We utilise {\sc Spark}'s machine-learning library, MLlib, and provide an interface from the standard variant data format, variant call format (VCF)~\cite{1KG2012}, which opens up the application of MLlib's different machine learning algorithms a wide range of genotype-based analysis tasks. 
To demonstrate its capability, we cluster variant datasets from the 1000 genomes project~\cite{1KG2012} to determine population structure using the k-means clustering algorithm available in MLlib's. 
In the first section we benchmark the performance and accuracy of MLlib's implementations against Hadoop's Mahout as well as more traditional methods (R, Python) limiting the used data to chromosome 22.
In section two we demonstrate {\sc Spark}'s full capacity by seamlessly scaling from 1\% to 100\% of the human genome.
In section three we replicate the analysis by using XX genomes from the Personal Genome Project~\cite{Lunshof2013}.
In the last section we discuss the pipeline for visualising the resulting cluster. 


\section*{Results}
\subsection*{{\sc Spark} enables faster clustering of individuals compared to traditional methods}

In this section we compare the time required to cluster individuals based on genomic variants using \variantSpark\ against ADAM as well as the more traditional approaches using Hadoop Mahout, Python and R. 
We limit the genomic variants to only chromosome 22 as the traditional approaches have substantially larger memory consumption rendering a whole-genome input infeasible.  
Furthermore, we perform the comparisons on a single virtual-machine to ensure the five different approaches have access to the same resources.
We use k-means clustering algorithm in the respective implementations, which all require the VCF input files to be pre-processed (see methods). 
We therefore compare the time required for the pre-processing step, as well as the time required for k-means clustering.
% DENIS: include this in the table too
%The python and R pre-processing implementation only runs on a single thread, so we don't include this in the comparison.

%We compare the time required to cluster individuals based on their chromosome 22 variants using various implementations. We compare our Variant{\sc Spark} implementation with ADAM and our Hadoop implementation, as well as a Python and R implementation.
%To keep the comparison fair for the Python and R implementations, we perform the comparisons on a single virtual-machine, rather than a cluster. Therefore, for k-means clustering, each implementation has access to the same resources.
%Each of the k-means clustering implementations that we compare require the input to be row vectors. We need to pre-process the VCF input files to this format, so we compare the time required for the pre-processing step, as well as the time required for k-means clustering.
%The python and R pre-processing implementation only runs on a single thread, so we don't include this in the comparison.


As shown in Table~\ref{fivewaycomparison}, pre-processing the data is fastest in \variantSpark\, requiring 2mins and 58secs. %178s 
This is almost 80\% faster than the ADAM implementation (12min and 48secs) %768s
or our Hadoop implementation(14min and 22secs). %862s
Unlike \variantSpark\ or our Hadoop implementation, which pre-process VCF files directly, ADAM requires the input to be in their ADAM file format. 
VCF files must hence be converted to this binary format before they can be pre-processed into the row vector format. 
Although this additional pre-processing step is only required as a one-off for each input file, it requires an additional 13mins and 13secs %793s
rendering our approach an order of magnitude faster. 
[Results for Python and R here].
This substantially different runtime for the pre-processing in Python and R is due to these implementations does not natively supporting multithreading while \variantSpark, ADAM and Hadoop using XX CPUs.

%Pre-processing the data is fastest in \variantSpark\, requiring 2mins and 58secs. Next fastest is the ADAM implementation, requiring 12min and 48secs. 
%Our Hadoop implementation took 14min and 22secs.
%Unlike Variant{\sc Spark} and our Hadoop implementation, which pre-process VCF files directly, ADAM requires the input to be in their ADAM file format. VCF files must be converted to this binary format before they can be pre-processed into the row vector format. Although this additional pre-processing step is only required as a one-off for each input file, it requires an additional 13mins and 13secs.

Clustering the samples 30\% fastest in \variantSpark\ compared to ADAM (1min and 20secs compared to 1min and 52secs) and  %80 vs 112
90\% faster than Hadoop (14mins and 23secs), see Table~\ref{fivewaycomparison}.
The similarity between \variantSpark\ and ADAM is expected, as they both use {\sc Spark}'s MLlib k-means implementation. 
The 30 second speed increase in \variantSpark\ is likely due us converting the VCF files to sparse vectors, whereas ADAM creates dense vectors.
%Due to the sparse nature of these row-vectors, the sparse vector format is likely to reduce some of the initial overhead.
We also compared the k-means clustering times to a multi-threaded approach in Python and R. 

Although an order of magnitude slower than \variantSpark\ and ADAM, k-means clustering in both Python and R were faster than our Hadoop implementation. Python required 11min and 29sec, whereas R required 7min and 25sec. 
This demonstrates one of Hadoop's disadvantages (one which {\sc Spark} dramatically improves on): the cost of disk IO. Where Hadoop's advantage is that it can scale to datasets that exceed memory size, in this example, the entire dataset can fit in memory on a single machine.
So, where the k-means implemtations we use in R and Python store the output from each k-means iteration in memory, Hadoop writes the output of each iteration to disk, whether or not the amount of available memory is sufficient to store this data.

This unnecessary disk IO forms a bottleneck that results in the longer run-times. {\sc Spark}, on the other hand, can cache the intermediate results in memory, virtually eliminating the disk IO bottleneck on smaller datasets.



%Clustering the samples is also fastest in \variantSpark\, albeit by a smaller margin. The time required is 1min and 20secs compared to 1min and 52secs, in Variant{\sc Spark} and ADAM, respectively. Hadoop however, required 14mins and 23secs.
%The similarity between Variant{\sc Spark} and ADAM is expected, as they both use {\sc Spark}'s MLlib k-means implementation. The 30 second speed increase in Variant{\sc Spark} is likely due to Variant{\sc Spark}'s pre-processing implementation converting VCF files to sparse vectors, whereas ADAM creates dense vectors.
%Due to the sparse nature of these row-vectors, the sparse vector format is likely to reduce some of the initial overhead.

%We also compared the k-means clustering times to a multi-threaded approach in Python and R. Although slower than the {\sc Spark} implementations in Variant{\sc Spark} and ADAM, k-means clustering in both Python and R were faster than our Hadoop implementation. Python required 11min and 29sec, whereas R required 7min and 25sec. This demonstrates one of Hadoop's disadvantages (one which {\sc Spark} dramatically improves on). Hadoop can scale to datasets that exceed memory size as wall as distributing work across nodes in a cluster. However, in this example, the entire dataset can fit in the memory on a single machine. The k-means implemtations
%we use in R and Python cluster the in-memory row vectors. Hadoop, however, writes the output of each MapReduce iteration to disk. This unnecessary disk IO forms a bottleneck that results in the longer run-times. {\sc Spark}, on the other hand, can cache the intermediate results in memory, virtually eliminating the disk IO bottleneck on smaller datasets.


%To summarise the total times for each big-data platform, Variant{\sc Spark} was 4mins and 18secs, ADAM was 14mins and 14secs (or 27mins and 53secs including the one-off conversion of VCF to ADAM) and our Hadoop implementation was 28mins and 45secs. 


\subsection*{{\sc VariantSpark} allows genome wide sampling of variants to improve clustering quality}

To demonstrate the scalability of {\sc Spark}, we continue to cluster the same individuals from the 1000 Genome Project, however, rather than clustering on a single chromosome, we cluster the individuals on different subsets of variants from their 22 autosomes, and finally every variant.
Pre-processing the data takes approximately 12 minutes for a 20\% sample. For 40\% of the genome, the time required is about 19 minutes, and finally, to preprocess 100\% of the genome, the time required is about 40 minutes.
For each of these sample sizes, the memory required doesn't exceed a modest 2GB per executor, even for the entire autosome. Memory requirements for clustering, however, increase by a 

Where our previous clustering of chromosome 22 required no more than 4GB memory per executor, clustering the entire genome increases this requirement to over 20GB.
One factor which results in these differing memory requirements is the sparsity of the vectors used to store the variants. Due to their sparse nature (the abundance of 0s) we can store them efficiently as Sparse Vectors, rather than Dense Vectors. However, to cluster these vectors, the k-means algorithm converts them to Dense Vectors.
With Dense Vectors, all the 0 values (which are excluded from Sparse Vectors so don't require any memory) are explicitly defined, considerably increasing the memory required.


\subsection*{Personal genomics project}



\subsection*{Graph visualization}
Add the image here \ldots


\section*{Methods}
\subsection*{Implementation}
For each of the implementations, we pre-process the VCF files to arrays of features.
In /VariantSpark/ we read in the VCF files as text files to a Resilient Data Set (RDD). RDDs allow us to process the files, line-by-line, in parallel. We parse each line as tab-separated values, and "zip" each value with it's heading from the VCF file.
The key-value pairs (KVPs) from each line in the VCF file are stored in separate arrays. We then "zip" each array with a unique index, where these indexes form identifiers for the alleles.
At this point, we have one array for each line from the VCF file. The values in the arrays are KVPs of the parsed tab-separated values zipped with their heading (individual ID), and each array has a unique index.
We now use a "flatMap" to convert these indexed arrays of KVPs into a less convoluted data-structure, as well as to filter out KVPs that aren't alleles (i.e. KVPs that are VCF metadata).
This results in KVPs where the key is the individual ID, and the value is a tuple of the allele and it's index. We effectively have one KVP for each allele from the VCF file. At this point, each allele is still a string straight from the VCF file (i.e. "0\textbar1:13"), however, we require doubles.
To convert this to a double, we take the Hamming distance of the allele, where a 1 represents a heterozygous variant, 2 a homozygous variant, and 0, no variant. Also, for each individual, we filter out any alleles that have no variant, leaving us with KVPs of variants (homozygous and heterozygous) only.
We can now group these KVPs by their key (the individual ID) and map them to sparse vectors. This results in a sparse vector for each individual, the format required by MLLibs machine learning algorithms.

For our ADAM comparison, we followed the ADAM implementation from (http://bdgenomics.org/blog/2015/02/02/scalable-genomes-clustering-with-adam-and-spark/).

 For our R implementation, we read in the 


%%%%%%%%%%%%%%%%%%%%%%%%%%%%%%%%%%%%%%%%%%%%%%
%%                                          %%
%% Backmatter begins here                   %%
%%                                          %%
%%%%%%%%%%%%%%%%%%%%%%%%%%%%%%%%%%%%%%%%%%%%%%

\begin{backmatter}

\section*{Competing interests}
  The authors declare that they have no competing interests.

\section*{Author's contributions}
    Text for this section \ldots

\section*{Acknowledgements}
  A.R.O was funded by the NSW Cancer Institute Big Data Big Impact schema, F.A.B by the National Health and Medical Research Council [1051757] and both D.C.B and N.S by Commonwealth Scientific and Industrial Research Organisation's Transformational Capability Platform, Science and Industry Endowment Fund and Information Management and Technology Services. The computation on Azure was funded by Microsoft Azure Research Award. 
The authors would like to thank Piotr Szul and Gareth Williams for their help with setting up Hadoop on the HPC system.

%%%%%%%%%%%%%%%%%%%%%%%%%%%%%%%%%%%%%%%%%%%%%%%%%%%%%%%%%%%%%
%%                  The Bibliography                       %%
%%                                                         %%
%%  Bmc_mathpys.bst  will be used to                       %%
%%  create a .BBL file for submission.                     %%
%%  After submission of the .TEX file,                     %%
%%  you will be prompted to submit your .BBL file.         %%
%%                                                         %%
%%                                                         %%
%%  Note that the displayed Bibliography will not          %%
%%  necessarily be rendered by Latex exactly as specified  %%
%%  in the online Instructions for Authors.                %%
%%                                                         %%
%%%%%%%%%%%%%%%%%%%%%%%%%%%%%%%%%%%%%%%%%%%%%%%%%%%%%%%%%%%%%

% if your bibliography is in bibtex format, use those commands:
\bibliographystyle{bmc-mathphys} % Style BST file (bmc-mathphys, vancouver, spbasic).
\bibliography{genotypeClustering}      % Bibliography file (usually '*.bib' )
% for author-year bibliography (bmc-mathphys or spbasic)
% a) write to bib file (bmc-mathphys only)
% @settings{label, options="nameyear"}
% b) uncomment next line
%\nocite{label}

% or include bibliography directly:
% \begin{thebibliography}
% \bibitem{b1}
% \end{thebibliography}

%%%%%%%%%%%%%%%%%%%%%%%%%%%%%%%%%%%
%%                               %%
%% Figures                       %%
%%                               %%
%% NB: this is for captions and  %%
%% Titles. All graphics must be  %%
%% submitted separately and NOT  %%
%% included in the Tex document  %%
%%                               %%
%%%%%%%%%%%%%%%%%%%%%%%%%%%%%%%%%%%

%%
%% Do not use \listoffigures as most will included as separate files

\section*{Figures}
  \begin{figure}[h!]
  \caption{\csentence{Sample figure title.}
      A short description of the figure content
      should go here.}
      \end{figure}

\begin{figure}[h!]
  \caption{\csentence{Sample figure title.}
      Figure legend text.}
      \end{figure}

%%%%%%%%%%%%%%%%%%%%%%%%%%%%%%%%%%%
%%                               %%
%% Tables                        %%
%%                               %%
%%%%%%%%%%%%%%%%%%%%%%%%%%%%%%%%%%%

%% Use of \listoftables is discouraged.
%%
\section*{Tables}
\label{fivewaycomparison}
\begin{table}[h!]
\caption{The resources consumption of the five compared methods as well as the accuracy measured as \ARI{} on chromosome 22.}
      \begin{tabular}{|c|ccc|ccc|c|c|}
        \hline
           Tool &  \multicolumn{3}{ c |}{Pre-processing} & \multicolumn{3}{ c |}{Clustering} & Accuracy \\
& threads & memory & time  &threads & memory & time  & \\
  \hline
\variantSpark\	& 8	& 32	& 2min 58sec	& 8	& 32	& 1min 20sec	& x	\\ 
ADAM		& 8	& 32	& 12min 48sec	& 8	& 32	& 1min 52sec	& x	\\
Hadoop		& 8	& 32	& 14min 22sec	& 8	& 32	& 14min 23sec	& x	\\
R			& 1	& 32	& 67min		& 8	& 32	& 7min 25sec	& x	\\
Python		& 1	& 32	& 92min		& 8	& 32	& 11min 29sec	& x	\\
  \hline
      \end{tabular}
\end{table}

%%%%%%%%%%%%%%%%%%%%%%%%%%%%%%%%%%%
%%                               %%
%% Additional Files              %%
%%                               %%
%%%%%%%%%%%%%%%%%%%%%%%%%%%%%%%%%%%

\section*{Additional Files}
  \subsection*{Additional file 1 --- Sample additional file title}
    Additional file descriptions text (including details of how to
    view the file, if it is in a non-standard format or the file extension).  This might
    refer to a multi-page table or a figure.

  \subsection*{Additional file 2 --- Sample additional file title}
    Additional file descriptions text.


\end{backmatter}
\end{document}
