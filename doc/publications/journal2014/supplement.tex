%% BioMed_Central_Tex_Template_v1.06
%%                                      %
%  bmc_article.tex            ver: 1.06 %
%                                       %

%%IMPORTANT: do not delete the first line of this template
%%It must be present to enable the BMC Submission system to
%%recognise this template!!

%%%%%%%%%%%%%%%%%%%%%%%%%%%%%%%%%%%%%%%%%
%%                                     %%
%%  LaTeX template for BioMed Central  %%
%%     journal article submissions     %%
%%                                     %%
%%          <8 June 2012>              %%
%%                                     %%
%%                                     %%
%%%%%%%%%%%%%%%%%%%%%%%%%%%%%%%%%%%%%%%%%

\newcommand{\mutrix}{mutagenetix}
\newcommand{\pheno}{phenomics}


%%%%%%%%%%%%%%%%%%%%%%%%%%%%%%%%%%%%%%%%%%%%%%%%%%%%%%%%%%%%%%%%%%%%%
%%                                                                 %%
%% For instructions on how to fill out this Tex template           %%
%% document please refer to Readme.html and the instructions for   %%
%% authors page on the biomed central website                      %%
%% http://www.biomedcentral.com/info/authors/                      %%
%%                                                                 %%
%% Please do not use \input{...} to include other tex files.       %%
%% Submit your LaTeX manuscript as one .tex document.              %%
%%                                                                 %%
%% All additional figures and files should be attached             %%
%% separately and not embedded in the \TeX\ document itself.       %%
%%                                                                 %%
%% BioMed Central currently use the MikTex distribution of         %%
%% TeX for Windows) of TeX and LaTeX.  This is available from      %%
%% http://www.miktex.org                                           %%
%%                                                                 %%
%%%%%%%%%%%%%%%%%%%%%%%%%%%%%%%%%%%%%%%%%%%%%%%%%%%%%%%%%%%%%%%%%%%%%

%%% additional documentclass options:
%  [doublespacing]
%  [linenumbers]   - put the line numbers on margins

%%% loading packages, author definitions

%\documentclass[twocolumn]{bmcart}% uncomment this for twocolumn layout and comment line below
\documentclass{bmcart}

%%% Load packages
%\usepackage{amsthm,amsmath}
%\RequirePackage{natbib}
%\RequirePackage{hyperref}
\usepackage[utf8]{inputenc} %unicode support
%\usepackage[applemac]{inputenc} %applemac support if unicode package fails
%\usepackage[latin1]{inputenc} %UNIX support if unicode package fails


%%%%%%%%%%%%%%%%%%%%%%%%%%%%%%%%%%%%%%%%%%%%%%%%%
%%                                             %%
%%  If you wish to display your graphics for   %%
%%  your own use using includegraphic or       %%
%%  includegraphics, then comment out the      %%
%%  following two lines of code.               %%
%%  NB: These line *must* be included when     %%
%%  submitting to BMC.                         %%
%%  All figure files must be submitted as      %%
%%  separate graphics through the BMC          %%
%%  submission process, not included in the    %%
%%  submitted article.                         %%
%%                                             %%
%%%%%%%%%%%%%%%%%%%%%%%%%%%%%%%%%%%%%%%%%%%%%%%%%


%\def\includegraphic{}
%\def\includegraphics{}
\usepackage{graphicx}
\usepackage{multirow}


%%% Put your definitions there:
\startlocaldefs
\endlocaldefs


%%% Begin ...
\begin{document}

\label{pgppops}
\begin{table}[h!]
\caption{Country of origin assignment to super-population.}
      \begin{tabular}{|c|c|c|c|}
\hline
Super Population		&	Super Population Size &	Populations	&	Population size \\
\hline
\multirow{4}{*}{AMR} 	&	\multirow{4}{*}{5}	&	Canada	&	2	\\
					&					&	Mexico	&	1	\\
					&					&	Puerto Rico               & 1               \\
					&					&	US Minor Outlying Islands & 1               \\
					\hline
\multirow{2}{*}{EAS}		&	\multirow{2}{*}{6}	&	China                     & 5               \\
					&					&	Taiwan, Province of China & 1	\\
					\hline
\multirow{14}{*}{EUR}	&	\multirow{14}{*}{20}	& Belgium				& 1	\\
					&					& Estonia				& 1	\\
					&					& Finland				& 1	\\
					&					& France				& 1	\\
					&					& Germany			& 1	\\
					&					& Greece				& 1	\\
					&					& Ireland				& 3	\\
					&					& Poland				& 2	\\
					&					& Russian Federation	& 1	\\
					&					& Slovenia			& 1	\\
					&					& Spain				& 2	\\
					&					& Sweden				& 1	\\
					&					& Ukraine				& 2	\\
					&					& United Kingdom		& 2	\\
					\hline
\multirow{2}{*}{SAS}		&	\multirow{2}{*}{4}	& Bangladesh			& 1	\\
					&					& India				& 3        \\     
					\hline
      \end{tabular}
\end{table}

\newpage

\section{R implementation comparison}

We compare our implementation using {\sc read.table} to one using the `VariantAnnotation' package. Reading the VCF file to a matrix with `readGT' is faster than reading a VCF file to a
table using R's `read.table', at 138 seconds compared to 208 seconds. However, the entire process takes around 30 minutes, so the relative savings in time are minimal.
One other difference is that `readGT' produces a matrix of variants, so we don't need to remove the first nine columns. However, this step takes less than one second.

The function to convert the variant strings to the Hamming distance consumes the majority of the time required for preprocessing. However, we created this as a vectorized function to
improve performance. Compared to our previous non-vectorized implementation, the run-time is halved.


\end{document}
