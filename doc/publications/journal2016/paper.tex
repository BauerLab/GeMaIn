\documentclass[11pt]{amsart}
\usepackage{geometry}                % See geometry.pdf to learn the layout options. There are lots.
\geometry{letterpaper}                   % ... or a4paper or a5paper or ... 
%\geometry{landscape}                % Activate for for rotated page geometry
%\usepackage[parfill]{parskip}    % Activate to begin paragraphs with an empty line rather than an indent
\usepackage{graphicx}
\usepackage{amssymb}
\usepackage{epstopdf}
\DeclareGraphicsRule{.tif}{png}{.png}{`convert #1 `dirname #1`/`basename #1 .tif`.png}

\title{Brief Article}
\author{The Author}
%\date{}                                           % Activate to display a given date or no date

\begin{document}
\maketitle
\section{Introduction}
As of 2014, obesity effects more than 600 million individuals worldwide, a number which has doubled since 1980 (who2015). The WHO definition for obesity is a BMI of greater than or equal to 30.
In addition to these individuals, 1.9 million adults are overweight.
Although changes in lifestyle factors, such as diet, contribute to this rise in obesity (Binkley2000), as a multifactorial disease (Grundy1998), certain genetic variants may predispose individuals to obesity.

Recently, Wheeler2013 identified 29 loci with evidence of association with obesity from an analysis of approximately 2 million SNPs.
This is in addition to 52 other genetic loci identified by GWAS over the four years prior (Loos2012).
However, with the relatively low coverage of GWAS compared to whole genome sequencing approaches, it's possible that GWAS studies may miss rare variants.
With genome sequencing becoming more accessible, the number of SNPs and other variants in recent datasets is increasing. For example,
the 1000 Genomes Project phase 3 dataset (Abecasis2012) contains more than 80 million variant sites.

As the size of the data increases, and the dimensionality (number of variants) increases, so does the compute complexity.
For this reason, we implemented MLlib, an interface between Variant Call Format (VCF) files and Spark.
Unlike approaches in R and Python, MLlib can scale further and is faster on equivalent, smaller datasets.

% What we do
Using Spark's Machine learning library, `Spark ml', we build a Random Forest model based on The Cancer Genome Atlas (TCGA) data, with the aim to
elucidate genetic variants associated with obesity.

We choose Random Forests for reasons such as their robustness to noise ( Breiman2001) 




%For example, individuals homozygous for risk alleles in in FTO,
%a transcription factor for IRX3, weigh on average 3 kg more than those homozygous for non-risk alleles.

%The decreasing cost of genome sequencing means we have more data 




%\subsection{dfh}



\end{document}  