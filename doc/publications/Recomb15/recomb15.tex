% This is LLNCS.DOC the documentation file of
% the LaTeX2e class from Springer-Verlag
% for Lecture Notes in Computer Science, version 2.4
\documentclass{llncs}
\usepackage{llncsdoc}
%

% Variable definitions
\newcommand{\variantSpark}{{\sc VariantSpark}}
\newcommand{\kMeans}{\textit{k}-means}
\newcommand{\ARI}{adjusted Rand index}



\begin{document}

\newcounter{save}\setcounter{save}{\value{section}}
{\def\addtocontents#1#2{}%
\def\addcontentsline#1#2#3{}%
\def\markboth#1#2{}%
%
\title{Using VariantSpark to learn population structure from genomic profiles}

\author{Aidan R. O'Brien\inst{1,2}, Jason Ross\inst{1}, Robert Dunne\inst{1}, Firoz Anwar\inst{1}, \and Denis C. Bauer\inst{1}}

\institute{
CSIRO, 11 Julius Av, 2113, Sydney, Australia \\
\and
School of Biomedical Sciences and Pharmacy, Faculty of Health, University of Newcastle, 2308, Newcastle, Australia \\
}

\maketitle
%
\begin{abstract}
This paragraph shall summarize the contents of the paper
in short terms.
\end{abstract}
%
\section{Introduction}
%

%TODO add a paper where ethnicity (i.e. allelefrequency) is important for drug dosage. 
Determining a persons ethnicity has become important for medical applications (EXAMPLES). This highlights the issue for clinical application where ancestry influences treatment (e.g. HLA allele genotyping from SNP information~\cite{Zheng2014}) and accurate population association may not be known for patients with diverse migrational background.


In previous work we have developed \variantSpark~\cite{OBrien}, a \kMeans{} approach using {\sc Spark} machine library, {\sc MLlib}, to cluster samples from the 1000 Genomes Project~\cite{1KG2012}.
%TODO add value of the kmeans method on the full dataset (the equivalent of the ARI=0.84)
We observed a perfect separation for the super populations African (AFR), East Asian (EAS) and European (EUR), however when adding the fourth super-population group, American (AMR), we observe that only a subset of the AMR individual form an independent cluster while the majority falls into the EUR cluster, resulting in an accuracy of 0.84, with -1 being random and 1 a perfect clustering.
As Spark enabled our approach to utilises whole genome information (40 Million variants), we argue that the analysis already includes a sufficient number of markers to separate populations with low genetic divergence~\cite{Patterson2006}. 

As \kMeans{}, does not take higher order interactions between features into account it is hence possible that the chosen methodology is not capable to accurately capture the underlying population structure. 
In fact, several studies have shown that single gene or single SNP approaches often fail to identify strong feature association or find marginal effect of the feature on the target phenotype~\cite{Bureau2005,Yoo2012,Qi2012}. 
As recently shown by Locke {\it et al.}~\cite{Locke2015} in a meta-study around genetic susceptibility of obesity only five of the 97 genomic loci have an independent association signal, the rest seem to jointly predict the metabolic phenotype.
Ethnicity may hence manifest itself in small additive multi-loci effects, known as epistasis~\cite{Mackay2014}, and hence requires to investigate the effect of one variant against all other variants in the feature vector.

Using conventional machine learning analysis to elucidate this additive effects is challenging as there are not enough samples to robustly fit the model over such a high-dimensional feature space~\cite{Chen2012}. 
As a result machine learning methods such as random forests (RF) have become popular as the underlying decision trees are build from a subsets of the features, which make the approach robust against overfitting issues~\cite{Breiman2001}. 
However, training an RF on such an expansive dataset as the 1000 genomes is computationally challenging and traditional massively parallel approaches to data processing are not easily applicable to machine learning tasks, as they iteratively refine models based on information from the full dataset. 

%TODO add paragraph about deep learning

We therefore use {\sc Spark} through or recently developed \variantSpark~\cite{OBrien} framework, which allows us to apply RF in the MLlib library to the variant file in the standard Variant Call Format (VCF) format. 
Please note that unlike {\sc Hadoop}, {\sc Spark} allows a more flexible software design to utilise node-parallelisation and offers in-memory cashing, both critical components for efficient ``big learning" tasks. 

To demonstrate \variantSpark's capability, we train an RF on the genomic profiles of the 1000 Genomes data using the annotated super-population labels. 
In the first section we determine the cross-validation accuracy of RF on three populations and compare against the unsupervised \kMeans{} clustering and the reported accuracy of a deep-learning approach using {\sc ADAM}~\cite{Massie2013}.
In section two we record the performance of RF and \kMeans{} on the full set of populations. 
In the last section we discuss the memory consumptions and runtime of RF compared to \kMeans{}.


%TODO check accuracy of titles
\section{Results}
%
\subsubsection{Logistic regression and captures the population better than deep learning in a three population test}
To test the quality We perform 5-fold cross-validation and report the mean accuracy over all training and testing folds along with the standard error.
We also perform 5 runs of \kMeans and report the mean Adjusted Rand Index (ARI) of the resulting models along with the standard error. 
We compare against the reported accuracy of the deep learning approach using the ADAM framework as reported on their webpage~\cite{Ferguson}. 
The analysis was performed on only three of the 14 populations, we hence restrict our comparison also to ASW (Americans of African Ancestry in SW USA), CHB (Han Chinese in Bejing, China), and GBR (British in England and Scotland).

We use the Adjusted Rand Index to test our \kMeans clustering quality as it allows us to compare the result to the expected clustering (based on populations) and takes into account the structure, rather than just the error-rate. We see an average ARI across five runs of \(0.772 \pm 0.086 \).
Although the average ARI is relatively low, it varies greatly between runs. The two best models clustered all but one and all but two individuals correctly, with an ARI of 0.988 and 0.978, respectively.
However, for the three less-accurate models, a common trait is that ASW and GBR individuals are grouped into one cluster with CHB individuals in a second cluster. The third cluster would contain between 1 and 10 seemingly random individuals.
One possible reason for this varied result is the cluster initialization step. Some individuals may have enough variation from the majority of their population, that if the algorithm uses one of their coordinates as an initial centroid, they will cluster separately from the rest of their population.

Although we see two of our five runs meet the accuracy of the deep-learning approach, where they achieved an accuracy of 99\% (or ARI of 0.975, with two misclassified individuals), our logistic regression approach demonstrates a consistent improvement.
With this approach, we achieve an accuracy of \(99.997 \pm 0.003\) across five folds.

To reiterate, these results are from the three populations, ASW, CHB and GBR, that belong to the super-populations, AFR, EAS and EUR, respectively.
As we mention in the introduction, we observed a perfect clustering result when we cluster the entirety of these three super-populations (rather than one population from each), with an ARI of 1.0. However, when we introduce the fourth super-population into the mix, AMR, the ARI drops to 0.84.
Given the perfect result from the logistic regression models on the three populations, and the improvement over \kMeans{}, we proceed with logistic regression on the four super-populations.  


\begin{table}
\caption{Cross-validated results, mean accuracy (and standard error) on 3 populations.}
\begin{center}
\renewcommand{\arraystretch}{1.4}
\setlength\tabcolsep{3pt}
\begin{tabular}{lcccc}
\hline\noalign{\smallskip}
method  & test folds & training folds & hold out \\
RF  & X (Y) & X (Y) & X (Y) \\
\kMeans & X (Y) & X (Y) & X (Y) \\
deep learning & X (Y) & X (Y) & X (Y) \\
\noalign{\smallskip}
\hline
\end{tabular}
\end{center}
\end{table}

\subsubsection{Logistic regression is more accurate than \kMeans{} in capturing the full super-population structure}
In this section we train and test our two methods on the full dataset of 14 populations and assess the accuracy with respect to the accurate classification into the 4 super-populations.
Using 10-fold cross-validation to assess the logistic regression classification results, we observe an average ARI of \(0.93 \pm{} XX\). This is an improvement on our previous result from \kMeans{} clustering the same dataset
where we observed an ARI of \(0.83\).
We also record the precision and recall for each of the groups, and observe that for each classification, and both values are always 1.0 for the EAS super-population. This demonstrates that each logistic regression model was able to classify every EAS individual correctly while not including any false positives in this group.
For the AMR group, however, we consistantly observed a relatively low precision (\(0.85\pm{} XX \)). This was generally due to individuals from the EUR super-population being misclassified as AMR. As a result, this led to EUR having a relatively low recall (\(0.9 \pm XX \)).
Although not present in the result of every fold, another observation is that AFR individuals would sometimes be misclassified as AMR. In all ten folds, we didn't observe AFR being classified as EUR, and vice-verse, nor AMR individuals being misclassified as another super-population.
These results demonstrate the diversity, or relative difference of some of the groups. 

%TODO [Discuss supervised vs unsupervised]
%TODO can you drill down to the population clusters and see if and where the best performing method still makes mistakes - this will help in the conclusion to argue that actually the labels are wrong rather than the method not being able to capture the structure (e.g. say if AMR are mixed in with GBR TSI)

\begin{table}
\caption{Cross-validated results, mean accuracy (and standard error) on the 4 super-populations.}
\begin{center}
\renewcommand{\arraystretch}{1.4}
\setlength\tabcolsep{3pt}
\begin{tabular}{lcccc}
\hline\noalign{\smallskip}
method  & test folds & training folds & hold out \\
RF  & X (Y) & X (Y) & X (Y) \\
\kMeans & X (Y) & X (Y) & X (Y) \\
\noalign{\smallskip}
\hline
\end{tabular}
\end{center}
\end{table}







\subsubsection{\kMeans{} clustering is faster and consumes less memory than random forests}

Although memory requirements scale linearly with 

%TODO [write up the observations and argue why one is less resource hungry than the other one]


\begin{table}
\caption{Runtime and memory consumption on the 4 super-populations}
\begin{center}
\renewcommand{\arraystretch}{1.4}
\setlength\tabcolsep{3pt}
\begin{tabular}{lcccc}
\hline\noalign{\smallskip}
method  & nodes & runtime & memory \\
RF  & X (Y) & X (Y) & X (Y) \\
\kMeans & X (Y) & X (Y) & X (Y) \\
\noalign{\smallskip}
\hline
\end{tabular}
\end{center}
\end{table}




\section{Conclusion}
In this paper we investigated the benefit of using a supervised method, random forests, compared to an unsupervised method,\kMeans, to determine the ethnicity from whole genome variant data. 
%TODO fill in the next sections
We find that X is better than Y because Z.
We also compare to the supervised deep learning approach as reported by Big Data Genomics using the ADAM framework and find X.
While we can improve the performance by X\% using the RF, even a supervised method is unable to accurately predict the assigned population labels. 
We therefore ague that the remaining mixture in X is reflective of the migrational background and hence the underlying genomic profile of Y is accurately similar to X.
%TODO add
This may be an important observation for EXAMPLES ABOVE.



\bibliographystyle{plain}
%TODO you need to find a way of shortening the Locket paper author list
\bibliography{recomb}  

\end{document}
